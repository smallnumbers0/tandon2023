\documentclass{article}
\usepackage{amsmath,amsthm,amssymb}
\renewcommand{\qedsymbol}{$\blacksquare$}
\usepackage{soul}
\usepackage[T1]{fontenc}
\usepackage{titling}
\usepackage[left=3cm, right=3cm, top=2cm]{geometry}
\setlength{\droptitle}{0 em} 
\title{\textbf{NYU Computer Science Bridge HW 11}\\
Summer 2023 Name: Jacky Choi}


\date{}
\begin{document}
\setul{}{2pt}
\maketitle

\noindent \textbf{Question 5}\\\\
\textbf{Exercise  a.}\\
Use mathematical induction to prove that for any positive integer n, 3 divide $n^3$ + 2n\\\\
BC: $n = 1$ : $1^3 + 2(1) = 3 = 3$\\
IH: Assume 3 divides $k^3 + 2(k)$ $3m = k^3 + 2(k)$\\
Show that 3 divides $(k+1)^3 + 2(k+1)$\\
Expanding gives us $k^3 + 2k + 3k^2 + 3k + 3$\\
Using Inductive Hypothesis gives us $3m + 3k^2 + 3k + 3$\\
Simplifying to $3(m + k^2 + k + 1)$\\
Since $(m + k^2 + k + 1)$ is an integer, we have shown that 3 divides $n^3+ 2n$\\\\
\textbf{Exercise  b.}\\
Use strong induction to prove that any positive integer $n (n\geq 2)$ can be written as a
product of primes\\\\
BC: n = 2. 2 is a prime number\\
IH: For $k \geq 2$. Assume that k can be represented as a product of primes.\\
Show that $k+1$ can be represented as a product of primes.\\
If $k+1$ is not prime, it is composite and can be written as a product of 2 primes\\
$2 \leq a,b \leq k+1$ where a and b are integers.\\
$\therefore$ Any positive integer $n \geq 2$ can be written as a product of primes.\\
\newpage

\noindent \textbf{Question 6}\\\\
\textbf{Exercise  7.4.1.a.}\\
$P(3)$ $1^2 + 2^2 + 3^2 = 3((3 + 1)(2 * 3 + 1))/6$\\\\
\textbf{Exercise  7.4.1.b.}\\
$P(k)$ $= (k(k + 1)(2 * k + 1))/6$\\\\
\textbf{Exercise  7.4.1.c.}\\
$P(k+1)$ $= (k+1(k + 2)(2k + 3))/6$\\\\
\textbf{Exercise  7.4.1.d.}\\
We must prove that P(1) is true\\\\
\textbf{Exercise  7.4.1.e.}\\
We must assume P(k) to be true and show that P(k+1) is true\\\\
\textbf{Exercise  7.4.1.f.}\\
The inductive hypothesis is P(k)\\\\
\textbf{Exercise  7.4.1.g.}\\
BC: $n = 1$ $\sum_{j=1}^{n} j^2 = \frac{n(n+1)(2n+1)}{6}$\\
$\sum_{j=1}^{1} 1^2 = \frac{1(1+1)(2(1)+1)}{6}$ is 1 = 1\\
IH: Assume P(k) $\sum_{j=1}^{k} j^2 = \frac{k(k+1)(2k+1)}{6}$ is true so P(k+1) is also true\\
$\sum_{j=1}^{k+1} j^2 = \sum_{j=1}^{k} j^2 + (k+1)^2$\\
Since $\sum_{j=1}^{k} j^2 = \frac{k(k+1)(2k+1)}{6}$ we can use the inductive hypothesis\\
Now we have $\frac{k(k+1)(2k+1)}{6} + (k+1)^2$\\
By Simplifying we will have $\frac{(k+1)(k+2)(2k+3)}{6}$\\
$\therefore$ P(k+1) is true\\\\
\textbf{Exercise  7.4.3c}\\\\
For $n\geq 1$ Prove $\sum_{j=1}^{n} \frac{1}{j^2} \leq 2-\frac{1}{n}$\\
BC: n = 1 $\frac{1}{1^2} = 2 - \frac{1}{1}$\\
IH: Assume $\sum_{j=1}^{k} \frac{1}{k^2} \leq 2 - \frac{1}{k}$\\
Show $\sum_{j=1}^{k+1} \frac{1}{(k+1)^2} \leq 2 - \frac{1}{k+1} + \frac{1}{(k+1)^2}$\\
$2 - \frac{1}{k+1} + \frac{1}{(k+1)^2}$\\
$2 - \frac{1}{k+1} + \frac{1}{k(k+1)}$\\
$2 - \frac{k}{k(k+1)} = 2 - \frac{1}{k+1}$\\
\textbf{Exercise  7.5.1a}\\\\
For $n > 0$ 4 evenly divides $3^{2n}-1$\\
BC: n = 1. $3^{2(1)}-1 = 8$ 8 is divisible by 4\\
IH: $p(k) = 3^{2k}-1$ $4m + 1= 3^{2k}$\\
Show $3^{2(k+1)}-1$ is divisible by 4\\
$3^(2k) * 3^2 - 1$\\
Using the inductive hypothesis $(4m+1) * 3^2 - 1$\\
$4(9m + 2)$ where 9m + 2 is an integer proving it is divisible by 4.

\end{document}