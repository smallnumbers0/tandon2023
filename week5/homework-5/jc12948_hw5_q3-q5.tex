\documentclass[11pt]{article}
\usepackage{amsmath,amsthm,amssymb}
\renewcommand{\qedsymbol}{$\blacksquare$}
\usepackage{soul}
\usepackage[T1]{fontenc}
\usepackage{titling}
\usepackage[left=3cm, right=3cm, top=2cm]{geometry}
\setlength{\droptitle}{0 em} 
\title{\textbf{NYU Computer Science Bridge HW5}\\
Summer 2023 Name: Jacky Choi}


\date{}
\begin{document}
\setul{}{2pt}
\maketitle

\noindent \textbf{Question 3}\\\\
\textbf{Exercise 4.1.3 b }\\
$f(x) = 1/(x^2 - 4)$\\
$\frac{1}{(x+2)(x-2)}$ shows this function is not defined on x = 2 and x = -2\\\\
\textbf{Exercise 4.1.3 c }\\
$f(x) = \sqrt{(x^2)}$\\
This is a well-defined function. The range is all $\mathbb{R} >= 0$ \\\\
\textbf{Exercise 4.1.5 b }\\
$Let A = \{2,3,4,5\}$\\
$f: A \rightarrow Z$ such that $f(x) = x^2$\\
Z = \{4, 9, 16, 25\}\\\\
\textbf{Exercise 4.1.5 d }\\
$f: \{0,1\}5 \rightarrow Z$. For $x \in {0,1}5$, f(x) is the number of 1's that occur in x. For example f(01101) = 3, because there are three 1's in the string "01101".\\
f(00000) = 0, f(00001) = 1, ... f(11111) = 5. Z = \{0,1,2,3,4,5\} since the max number of 1s can appear is 5 and min is 0.\\\\
\textbf{Exercise 4.1.5 h }\\
$Let A = \{1,2,3\}$\\
$f: A x A \rightarrow \mathbb{Z} x \mathbb{Z}$ where $f(x,y) = (y,x)$\\
= \{(1,1), (1,2), (1,3), (2,1), (2,2), (2,3), (3,1), (3,2), (3,3)\}\\
\textbf{Exercise 4.1.5 i }\\
$Let A = \{1,2,3\}$\\
$f: A x A \rightarrow \mathbb{Z} x \mathbb{Z}$ where $f(x,y) = (x, y+1)$\\
\{  (1,2), (1,3), (1,4), (2,2), (2,3), (2,4),  (3,2), (3,3), (3,4) \}\\\\
\textbf{Exercise 4.1.5 l }\\
$Let A = \{1,2,3\}$\\
$f: P(A) \rightarrow P(A)$ For $X \subseteq A, f(X) = X - \{1\}$\\
\{ \{\}, \{2\}, \{3\}, \{2,3\} \}\\\\

\newpage

\noindent \textbf{Question 4 Part I}\\\\
\textbf{Exercise 4.2.2 c }\\
$h: \mathbb{Z} \rightarrow \mathbb{Z}. h(x) = x^3$\\
h(x) cannot = 5 One to one but not onto.\\\\
\textbf{Exercise 4.2.2 g }\\
$f: \mathbb{Z} x \mathbb{Z} \rightarrow \mathbb{Z} x \mathbb{Z} . f(x,y) = (x + 1, 2y)$\\
f(x,y) cannot = (0,5) One to one but no onto.\\
\textbf{Exercise 4.2.2 k }\\
$f: \mathbb{Z}^+ x \mathbb{Z}^+ \rightarrow \mathbb{Z}^+ , f(x,y) = (2^x+y)$\\
Neither one to one or onto. f(1,4) and f(2,2) both = 6. f(x,y) cannot = 1 \\\\
\textbf{Exercise 4.2.4 b }\\
$f: 0, 13 \rightarrow 13$. The output of f is obtained by taking the input string and replacing the first bit by 1, regardless of whether the first bit is a 0 or 1. For example, $f(001) = 101$ and $f(110) = 110$.\\\\
Neither one to one or onto. f(010) and f(110) = 110 and f(x) also cannot be 000\\\\
\textbf{Exercise 4.2.4 c }\\\\
$f: 0, 13 \rightarrow 13$. The output of f is obtained by taking the input string and reversing the bits. For example $f(011) = 110$.\\\\
Both one to one and onto. \\\\
\textbf{Exercise 4.2.4 d }\\
$f: \{0,1\}^3 \rightarrow \{0,4\}^4$. The output of f is obtained by taking the input string and adding an extra copy of the first bit to the end of the string. For example, $f(100) = 1001.$\\\\
One to one, but not onto. Adding first bit doesn't allow us to hit 1000 f(100) = 1001\\\\
\textbf{Exercise 4.2.4 g }\\
Let A be defined to be the set \{1,2,3,4,5,6,7,8\} and let B = \{1\}. $f: P(A) \rightarrow P(A)$. For $X \subseteq A, f(X) = X - B$. Recall that for a finite set A, P(A) denotes the power set of A which is the set of all subsets of A.\\\\
Neither one to one or onto. If X1 = \{1,2,3\} and X2 = \{2,3\} then they are both \{2,3\} f(X) also cannot be \{1\}\\\\
\newpage

\noindent \textbf{Question 4 Part II}\\\\
\textbf{Give an example of a function from the set of integers to the set of positive integers that is:}\\
\textbf{a. one-to-one but not onto }\\\\
$f(z) = |x| + 5$\\\\
\textbf{b. onto but not one to one }\\\\
$f(x) = x^2$\\\\
\textbf{c. one-to-one and onto }\\\\
$f(x) = x + 1$\\\\
\textbf{d. neither one-to-one nor onto }\\\\
$f(x) = 5$\\\\

\newpage

\noindent \textbf{Question 5}\\\\
\textbf{Exercise 4.3.2 c}\\
$f: \mathbb{R} \rightarrow \mathbb{R}. f(x) = 2x + 3$\\\\
$f^{-1}(x) = \frac{x-3}{2}$\\\\
\textbf{Exercise 4.3.2 d}\\
Let A be defined to the set \{1,2,3,4,5,6,7,8\}\\
$f: P(A) \rightarrow \{0,1,2,3,4,5,6,7,8\}$\\
For $X \subseteq |X|$\\\\
Not a well defined function. It is not one to one as they can map to a single output.\\\\
\textbf{Exercise 4.3.2 g}\\
$f: \{0,1\}^3 \rightarrow \{0,1\}^3$\\
$f^{-1}(x) = f(x)$=\\  
$\{0,1\}^3 = \{0,1\}^3$\\\\
\textbf{Exercise 4.3.2 i}\\
$f: \mathbb{Z} x \mathbb{Z} \rightarrow \mathbb{Z} x \mathbb{Z}, f(x,y) = (x + 5, y - 2)$ \\\\
$f^{-1}(x,y) = (x-5, y+2)$\\\\
\newpage

\noindent\textbf{Exercise 4.4.8}\\
$f(x) = 2x + 3$ $g(x) = 5x + 7$ $h(x) = x^2 + 1$ \\\\
\textbf{Exercise 4.4.8 c}\\
$f \circ h$ \\\\
$f \circ h(x) = 2x^2 + 5$\\\\
\textbf{Exercise 4.4.8 d}\\
$h \circ f$ \\\\
=$((2x + 3)^2 + 1)$\\
$h \circ f(x) = 4x^2 + 12x + 10$ \\\\
\textbf{Exercise 4.4.2}\\
$f(x) = x^2$ $g(x) = 2^x$ $h(x) = \lceil \frac{x}{5} \rceil$\\\\
\textbf{Exercise 4.4.2 b}\\
Evaluate $(f \circ h)(52)$\\\\
$f \circ h(52) = (\lceil \frac{52}{5} \rceil)^2  $\\
=$11^2$ = 121\\\\
\textbf{Exercise 4.4.2 c}\\
Evaluate $(g \circ h \circ f)(4)$\\
f(4) = 16\\
=$g \circ h(16) = g(4) = 16$ \\
\textbf{Exercise 4.4.2 d}\\
Give mathematical expression for h o f.\\\\
$h \circ f = \lceil \frac{x^2}{5} \rceil$\\\\
\textbf{Exercise 4.4.6 c}\\
What is (h o f)(010)?\\\\
$h \circ f(010)$ = h(110) = 111\\\\
\textbf{Exercise 4.4.6 d}\\
Range of h o f: \{101, 111\}\\\\
\textbf{Exercise 4.4.6 e}\\\\
Range of g o f: \{001, 011, 101, 111\} \\\\
\newpage
\noindent \textbf{Extra Credit:}\\\\
\textbf{Exercise 4.4.4 c}\\
Is it possible that f is not one-to-one and g o f is one-to-one? Justify your answer. If the answer is "yes", give a specific example for f and g. \\\\
No.

\begin{proof}
If f is not one to one, then $g \circ f$ must not be one to one\\
There exists an $x_1$ and $x_2$ $\in X$ such that $x_1 \not{=} x_2 \wedge f(x_1) = f(x_2)$ and $g(f(x_1)) = g(f(x_2))$. And since $x_1 \not{=} x_2$\\
$\therefore g \circ f$ is not one to one.\\
\end{proof}
\noindent \textbf{Exercise 4.4.4 d}\\
Is it possible that g is not one-to-one and g o f is one-to-one? Justify your answer. If the answer is "yes", give a specific example for f and g.\\\\
yes\\
\begin{proof}
If $f(x_1)$ = a and g(a) = 1 and $f(x_2$ = b and g(b) = 2, there exists a c such that f(x) is not one to one but g(f(c) = 2. Since g(f(c) = 2 and g(b) = 2\\
$\therefore g \circ f$ is one to one\\
\\
\end{proof}
\end{document}