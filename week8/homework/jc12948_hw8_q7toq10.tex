\documentclass{article}
\usepackage{amsmath,amsthm,amssymb}
\renewcommand{\qedsymbol}{$\blacksquare$}
\usepackage{soul}
\usepackage[T1]{fontenc}
\usepackage{titling}
\usepackage[left=3cm, right=3cm, top=2cm]{geometry}
\setlength{\droptitle}{0 em} 
\title{\textbf{NYU Computer Science Bridge HW8}\\
Summer 2023 Name: Jacky Choi}


\date{}
\begin{document}
\setul{}{2pt}
\maketitle

\noindent \textbf{Question 7}\\\\
\textbf{Exercise 6.1.5 b}\\
Three of a kind\\
$(C(13,1) * C(4,3) * C(12,2) * C(4,1) * C(4,1)) / C(52,5) \approx 0.021$\\\\
\textbf{Exercise 6.1.5 c}\\
5 cards of same suit\\
$C(4,1) * C(13,5) / C(52,5) \approx 0.00198$\\\\
\textbf{Exercise 6.1.5 d}\\
Two of a kind\\
$(C(13,1) * C(4,2) * C(12,3) * C(4,1) * C(4,1) * C(4,1)) / C(52,5) \approx 0.4225$\\\\
\textbf{Exercise 6.2.4 a}\\
The hand has at least 1 club\\
Subtract the probability of all hands without a club.\\
$1 - C(39,5) / C(52,5) \approx 0.778$\\\\
\textbf{Exercise 6.2.4 b}\\
The hand has at least two cards with the same rank\\
Subtract the probability of all hands without 2 cards in the same rank\\
$1 - (C(13,5) * 4^5 / C(52,5)) \approx 0.492$\\\\
\textbf{Exercise 6.2.4 c}\\
The hand has exactly one club or one spade\\
$(C(13,1)*C(39,4)/C(52,5) * 2) - (2*C(13,1)*C(26,3)/C(52,5)) \approx 0.654$\\\\
\textbf{Exercise 6.2.4 d}\\
The hand has at least one club or at least one spade\\
$1 - (C(13+13,5)/C(52,5)) \approx 0.975$\\\\
\newpage
\noindent \textbf{Question 8}\\\\
\textbf{Exercise 6.3.2 a}\\
p(A) = 6! / 7! = 1/7\\
p(B) = (7! / 2!) / 7! = 1/2\\
|a,b,c,def,g| = 5, p(C) 5! / 7! = 1/42\\\\
\textbf{Exercise 6.3.2 b}\\
p(A|C) = 2! * 3! / 5! = 1/10\\\\
\textbf{Exercise 6.3.2 c}\\
p(B|C) = 1/2\\\\
\textbf{Exercise 6.3.2 d}\\
p(A|B) = 1/7\\\\
\textbf{Exercise 6.3.2 e}\\\\
Which pairs of events among A, B, and C are independent?\\
A,B and B,C are independent p(A) = p(A|B) and p(B) = p(B|C)\\\\
\textbf{Exercise 6.3.6 b}\\
The first 5 flips comes up heads. The last 5 flips comes up tails\\
$(1/3)^5 * (2/3)^5$\\\\
\textbf{Exercise 6.3.6 c}\\
The first flip comes up heads. The rest of the flips come up tails\\
$(1/3)^1 * (2/3)^9$\\\\
\textbf{Exercise 6.4.2 d}\\\\
Let F be the event that we chose the fair die. \\
Let R be the event that rolling the dice six times gives 4, 3, 6, 6, 5, 5.\\
$p(F) = \frac{1}{2}$\\
$p(\overline{F}) = \frac{1}{2}$\\
$p(R|F) = \frac{1}{6}$\\
$p(R|\overline{F}) = 0.15^4 * 0.25^2 * \frac{1}{2}$\\\\
$p(R|F) = \frac{p(R|F)p(F)}{p(R|F)p(F)+p(R|\overline{F}p(\overline{F}))}$\\
$p(R|F) = \frac{\frac{1}{6}^6 * \frac{1}{2}}{\frac{1}{6}^6 * \frac{1}{2} + 0.15^4 * 0.25^2 * \frac{1}{2}} \approx 0.40$

\newpage
\noindent \textbf{Question 9}\\\\
\textbf{Exercise 6.5.2 a}\\
A hand of 5 cards is dealt from a perfectly shuffled deck of playing cards. Let the random variable A denote the number of aces in the hand.\\
Range of A = \{0,1,2,3,4\}\\\\
\textbf{Exercise 6.5.2 b}\\
Distribution over random Variable A\\
$\{(0,(C48,5)/C(52,5)),\\
(1,C(4,1)*C(48,4)/C(52,5)),\\
(2,C(4,2)*C(48,3)/C(52,5)),\\
(3,C(4,3)*C(48,2)/C(52,5)),\\
(4C(4,4)*C(48,1)/C(52,5))\}$\\\\
\textbf{Exercise 6.6.1 a}\\
Two student council representatives are chosen at random from a group of 7 girls and 3 boys. Let G be the random variable denoting the number of girls chosen. What is E[G]?\\\\
$E[G] = C(7,1)*C(3,1)/C(10,2) + 2(C(7,2)/C(10,2)) = 1.4$\\\\
\textbf{Exercise 6.6.4 a}\\
A fair die is rolled once. Let X be the random variable that denotes the square of the number that shows up on the die. For example, if the die comes up 5, then X = 25. What is E[X]?\\
$E[X] = (1^2 * 1/6) + (2^2 * 1/6) + (3^2 * 1/6) + (4^2 * 1/6) + (5^2 * 1/6) + (6^2 * 1/6) \approx 15.167$\\\\
\textbf{Exercise 6.6.4 b}\\\\
A fair coin is tossed three times. Let Y be the random variable that denotes the square of the number of heads. For example, in the outcome HTH, there are two heads and Y = 4. What is E[Y]?\\
\{HHH, HHT, HTT, HTH, ... TTT\}\\
$E[Y] = (0^2 * 1/8) + (1^2 *3/8)+ (2^2 *3/8)+ (3^2 *1/8) = 3$ \\\\
\textbf{Exercise 6.7.4 a}\\\\
A class of 10 students hang up their coats when they arrive at school. Just before recess, the teacher hands one coat selected at random to each child. What is the expected number of children who get his or her own coat?\\
Let X denote the number of children who get a coat.\\
Don't all children still get a coat? 1/10\\
$E[X] = E[X_1] + E[X_2] + E[X_3] + E[X_4] + E[X_5] + E[X_6] + E[X_7] + E[X_8] + E[X_9] + E[X_10] = 1$
\newpage
\noindent \textbf{Question 10}\\\\
\textbf{Exercise 6.8.1 a}\\
The probability that a circuit board produced by a particular manufacturer has a defect is 1\%. You can assume that errors are independent, so the event that one circuit board has a defect is independent of whether a different circuit board has a defect\\
What is the probability that out of 100 circuit boards made exactly 2 have defects?\\
1\% chance of defect\\
$C(100,98) * 0.99^{98}* 0.01^2 \approx  0.185$\\\\
\textbf{Exercise 6.8.1 b}\\
What is the probability that out of 100 circuit boards made at least 2 have defects?\\
$1 - (0.99^{100} + 0.99^{99}) \approx 0.264$\\\\
\textbf{Exercise 6.8.1 c}\\
What is the expected number of circuit boards with defects out of the 100 made?\\
$100 * 0.01 = 1$\\\\
\textbf{Exercise 6.8.1 d}\\
Now suppose that the circuit boards are made in batches of two. Either both circuit boards in a batch have a defect or they are both free of defects. The probability that a batch has a defect is 1\%. What is the probability that out of 100 circuit boards (50 batches) at least 2 have defects? What is the expected number of circuit boards with defects out of the 100 made? How do your answers compared to the situation in which each circuit board is made separately?\\\\
Probability: $1- (C(50,50) * 0.01^0 * 0.99^{50} + C(50,49) * 0.99^{49} * 0.01^1) \approx 0.089$\\
Expected: $50 * 0.01 = 0.5$\\
The amount of defects is decreased if made in batches than compared by making each seperately.\\\\
\textbf{Exercise 6.8.3 b}\\
A gambler has a coin which is either fair (equal probability heads or tails) or is biased with a probability of heads equal to 0.3. Without knowing which coin he is using, you ask him to flip the coin 10 times. If the number of heads is at least 4, you conclude that the coin is fair. If the number of heads is less than 4, you conclude that the coin is biased.\\
What is the probability that you reach an incorrect conclusion if the coin is biased?\\\\
$1 - (C(10,0) + 0.3^0 + 0.7^{10}) + (C(10,1) + 0.3^1 + 0.7^9) + (C(10,2) + 0.3^2 + 0.7^8) + (C(10,3) + 0.3^3 + 0.7^7) \approx 0.35$


\end{document}