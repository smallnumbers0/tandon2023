\documentclass[11pt]{article}
\usepackage{amsmath,amsthm,amssymb}
\usepackage{soul}
\usepackage[T1]{fontenc}
\usepackage{titling}
\usepackage[left=3cm, right=3cm, top=2cm]{geometry}
\setlength{\droptitle}{0 em} 
\title{\textbf{NYU Computer Science Bridge HW2}\\
Summer 2023 Name: Jacky Choi}


\date{}
\begin{document}
\setul{}{2pt}

\maketitle

\noindent Question 5:\\
a) Solve the following questions from the Discrete Math zyBook:\\
1. Exercise 1.12.2, sections b, e\\\\
b. 
\begin{center}
  \begin{tabular}{l}
    $p \rightarrow (q \wedge r)$\\
  	$\neg q$\\
	\hline
   $ \therefore \neg p$\\
  \end{tabular} \\
 \begin{center}
  \begin{tabular}{lclcl}
1.& $\neg q$ & Hypothesis\\
2.& $\neg q \vee \neg r$ & Addition, 1\\
3.& $\neg (q \wedge r)$ & De Morgan, 2\\
4.& $ p \rightarrow (q \wedge r) $ & Hypothesis\\
5.& $\neg p$& Modus tollens, 3 4\\

  \end{tabular}
\end{center}
\end{center}

\noindent e. 
\begin{center}
  \begin{tabular}{l}
    $p \vee q$\\
  $ \neg p \vee r$\\
    $  \neg q$\\
    \hline
   $ \therefore r$\\
  \end{tabular}\\
  \begin{center}
   \begin{tabular}{lclcl}
1.& $p \vee q$ & Hypothesis\\
2.& $\neg p \vee r$ & Hypothesis\\
3.& $q \vee r$ &Resolution, 1 2\\
4.& $ \neg q$ & Hypothesis\\
5.& $r$& Disjunctive Syllogism, 3 4\\
  \end{tabular}
\end{center}
\end{center}
\pagebreak
\noindent Exercise 1.12.3, section c\\

\begin{center}

  \begin{tabular}{l}
   $ p \vee q$\\
    $\neg p $\\
    \hline
    $\therefore q $\\
  \end{tabular}\\
  
  \begin{center}
   \begin{tabular}{lclcl}
   

1.& $p \vee q$ & Hypothesis\\
2.& $\neg (\neg p) \vee q$ &Double Negation \\
3.& $\neg p \rightarrow q$ &Conditional Identity, 2\\
4.& $ \neg p$ & Hypothesis\\
5.& $q$& Modus Ponens, 3 4\\


  \end{tabular}
  \end{center}
  
\end{center}

\noindent Exercise 1.12.5, section c
\begin{center}
  \begin{tabular}{l}
   \text{I will buy a new car and a new house only if I get a job.}\\
   \text{I am not going to get a job.}\\
   \hline
  $ \therefore \text{I will not buy a new car.}$
  \end{tabular}
  
  \begin{center}
  c: I will buy a new car\\
  h: I will buy a new house\\
  j: I get a job\\
  \end{center}

   \begin{tabular}{l}
   $ (c \wedge h) \rightarrow j $\\
   $\neg j$\\
   \hline
    $\therefore \neg c $\\
  \end{tabular}\\
\end{center}

\noindent Argument is not valid. When both hypotheses are true the conclusion is false: c = True, h and j = False \\

\noindent Exercise 1.12.5, section d
\begin{center}
  \begin{tabular}{l}
   \text{I will buy a new car and a new house only if I get a job.}\\
   \text{I am not going to get a job.}\\
   I will buy a new house.\\
   \hline
  $ \therefore \text{I will not buy a new car.}$
  \end{tabular}
  
  \begin{center}
  c: I will buy a new car\\
  h: I will buy a new house\\
  j: I get a job\\
  \end{center}
  
   \begin{tabular}{l}
   $ j \rightarrow (c \wedge h) $\\
   $\neg j$\\
    $ h $\\
    \hline
    $\therefore \neg c $\\
  \end{tabular}\\

  
\end{center}

\noindent The argument is valid

\begin{center}
\begin{tabular}{lclcl}
   

1.& $(c \wedge h) \rightarrow j$ & Hypothesis\\
2.& $\neg (c\wedge h) \vee j$ &Conditional Identity\\
3.& $j \vee \neg (c \wedge h) $ &Commutative law, 2\\
4.& $ \neg j$ & Hypothesis\\
5.& $\neg(c \wedge h)$& Disjunctive Syllogism, 3 4\\
6.& $\neg c \vee \neg h$ & De Morgans, 5\\
7.& $\neg h \vee \neg c$ &Commutative\\
8.& $h$ & Hypothesis\\
9.& $ \neg \neg h$ & Double Negation\\
10.& $\neg c$& Disjunctive Syllogism, 7 9\\

\end{tabular}
\end{center}


\noindent b)  \\
Exercise 1.13.3, section b\\

\begin{center}
  \begin{tabular}{l}
  $\exists x (P(x) \vee Q(x))$\\
  $\exists x \neg Q(x)$\\
     \hline
  $ \therefore \exists x \neg P(x) $
 

  \end{tabular}
\end{center}

\begin{center}
\begin{tabular}{|l|l|l|}
\hline
 & P& Q \\ \hline
a & F & T \\ \hline
b & F & F \\ \hline
\end{tabular}
\end{center}

\noindent When x = a$\exists x (P(x) \vee Q(x))$ is true and when x = b $\exists x \neg Q(b)$ is also true, but $\exists xP(x)$is false proving it is not valid\\


\noindent Exercise 1.13.5, section d
\begin{center}
  \begin{tabular}{l}
  Every student who missed class got a detention.\\
  Penelope is a student in the class.\\
  Penelope did not miss class.\\
    \hline
  $ \therefore \text{Penelope did not get a detention}$
  \end{tabular}
  
  \begin{center}
\begin{tabular}{l}

M(x): x missed class\\
D(x); x got a detention\\
\end{tabular}
\end{center}

\begin{center}
\begin{tabular}{l}
$\forall x (M(x) \rightarrow D(x))$\\
Penelope is a student in the class\\
$\neg M(Penelope)$\\
\hline
$\therefore \neg D(Penelope)$\\
\end{tabular}
\end{center}

\noindent The argument is invalid. If M(Penelope) is false and D(Penelope) is true, then the hypothesis are true and conclusion is false.\\
\end{center}

Exercise 1.13.5, section e
\begin{center}
  \begin{tabular}{l}
  Every student who missed class or got a detention did not get an A.\\
  Penelope is a student in the class.\\
  Penelope got an A.\\
   \hline
  $ \therefore \text{Penelope did not get a detention}$
 \end{tabular}
\end{center}

  \begin{center}
\begin{tabular}{l}

M(x): x missed class\\
A(x): x got an A\\
D(x); x got a detention\\
\end{tabular}
\end{center}

\begin{center}
\begin{tabular}{l}
$\forall x (M(x) \vee D(x)) \rightarrow A(x)$\\
Penelope is a student in the class\\
$A(Penelope)$\\
\hline
$\therefore \neg D(Penelope)$\\
\end{tabular}
\end{center}

\begin{center}
\begin{tabular}{lclcl}
   

1.& Penelope is a student in the class & Hypothesis\\
2.& $\forall x (M(x) \vee D(x) \rightarrow \neg A(x))$ &Hypothesis\\
3.& $M(Penelope) \vee D(Penelope) \rightarrow \neg A(Penelope) $ &Universal Insatntiation\\
4.& $ A(Penelope)$ & Hypothesis\\
5.& $\neg (\neg A(Penelope))$& Double negation, 4\\
6.& $\neg (M(Penelope) \vee D(Penelope))$ & Modus Tollens, 3 5\\
7.& $\neg M(Penelope) \wedge \neg D(Penelope)$ & De Morgan\\
8.& $\neg D(Penelope) \wedge \neg M(Penelope)$ & cumutative\\
9.& $\neg D(Penelope)$ & simplification\\

\end{tabular}
\end{center}


\noindent Question 6:\\
Exercise 2.4.1, section d\\
The product of two odd integers is an odd integer.
\begin{proof}
	Let x, y be an odd integer
	\begin{align*}
	\end{align*}
\end{proof}

Exercise 2.4.3, section b\\
If x is a real number and $x \leq 3, \text{then} 12 - 7x + x^2 \geq 0$.\\
\begin{proof}
	Let x be a real number$ \wedge x \leq 3$
\end{proof}



\noindent Question 7:\\


Exercise 2.5.1 section d \\

For every integer n, if $n^2 - 2n + 7$ is even, then n is odd\\
\begin{proof}
  Let $t,u \in \mathbb{R}$ where $t=xy$ and $u=zw$. So,
  \begin{align*}
    4xyzw &= 2\cdot2tu \\
    &\le 2\cdot(t^2+u^2) \\
    &= 2\cdot((xy)^2+(zw)^2) &&\text{(substituting variables)} \\
    &= 2\cdot(x^2y^2+z^2w^2) \\ 
    &= 2x^2y^2+2z^2w^2 \\
    &\le ((x^2)^2+(y^2)^2)+((z^2)^2)+(w^2)^2) \\
    &= x^4+y^4+z^4+w^4 &&\qedhere
  \end{align*}
\end{proof}

Exercise 2.5.4 section a\\

For every pair of real numbers x and y, if $x^3 + xy^2 \leq x^2y + y^3$ then $x \leq y$\\

Exercise 2.5.4 section b\\

For every pair of real numbers x and y, if $x + y > 20$\\
  
Exercise 2.5.5 section c\\

For every non-zero real number x, if x is irrational then $1/x$ is irrational\\

\noindent Question 8: \\


Exercise 2.6.6 sections c\\

The average of three real numbers is greater than or equal to at least one of the numbers.\\

Exercise 2.6.6 sections d\\

There is no smallest integer\\

\noindent Question 9:\\
Exercise 2.7.2 section b\\

If integers x and y have the same parity, then $x + y$ is even. The parity of a number tells whether the number is odd or even. If x and y have the same parity, they are either both even or both odd.\\


\end{document}