\documentclass[11pt]{article}
\usepackage{amsmath,amsthm,amssymb}
\renewcommand{\qedsymbol}{$\blacksquare$}
\usepackage{soul}
\usepackage[T1]{fontenc}
\usepackage{titling}
\usepackage[left=3cm, right=3cm, top=2cm]{geometry}
\setlength{\droptitle}{0 em} 
\title{\textbf{NYU Computer Science Bridge HW2}\\
Summer 2023 Name: Jacky Choi}


\date{}
\begin{document}
\setul{}{2pt}

\maketitle

\noindent \textbf{Question 5:}\\\\
\textbf{Exercise 1.12.2 B }
\begin{center}
  \begin{tabular}{l}
    $p \rightarrow (q \wedge r)$\\
  	$\neg q$\\
	\hline
   $ \therefore \neg p$\\
  \end{tabular} \\
 \begin{center}
  \begin{tabular}{lclcl}
1.& $\neg q$ & Hypothesis\\
2.& $\neg q \vee \neg r$ & Addition, 1\\
3.& $\neg (q \wedge r)$ & De Morgan, 2\\
4.& $ p \rightarrow (q \wedge r) $ & Hypothesis\\
5.& $\neg p$& Modus tollens, 3 4\\ 

  \end{tabular}
\end{center}
\end{center}
\qed 


\noindent \textbf{Exercise 1.12.2 E}
\begin{center}
  \begin{tabular}{l}
    $p \vee q$\\
  $ \neg p \vee r$\\
    $  \neg q$\\
    \hline
   $ \therefore r$\\
  \end{tabular}\\
  \begin{center}
   \begin{tabular}{lclcl}
1.& $p \vee q$ & Hypothesis\\
2.& $\neg p \vee r$ & Hypothesis\\
3.& $q \vee r$ &Resolution, 1 2\\
4.& $ \neg q$ & Hypothesis\\
5.& $r$& Disjunctive Syllogism, 3 4\\
  \end{tabular}
\end{center}
\end{center}
\qed
\pagebreak\\
\noindent \textbf{Exercise 1.12.3 C}\\

\begin{center}

  \begin{tabular}{l}
   $ p \vee q$\\
    $\neg p $\\
    \hline
    $\therefore q $\\
  \end{tabular}\\
  
  \begin{center}
   \begin{tabular}{lclcl}
   

1.& $p \vee q$ & Hypothesis\\
2.& $\neg (\neg p) \vee q$ &Double Negation \\
3.& $\neg p \rightarrow q$ &Conditional Identity, 2\\
4.& $ \neg p$ & Hypothesis\\
5.& $q$& Modus Ponens, 3 4\\


  \end{tabular}
  \end{center}
  
\end{center}
\qed

\noindent \textbf{Exercise 1.12.5 C}
\begin{center}
  \begin{tabular}{l}
   \text{I will buy a new car and a new house only if I get a job.}\\
   \text{I am not going to get a job.}\\
   \hline
  $ \therefore \text{I will not buy a new car.}$
  \end{tabular}
  
  \begin{center}
  c: I will buy a new car\\
  h: I will buy a new house\\
  j: I get a job\\
  \end{center}

   \begin{tabular}{l}
   $ (c \wedge h) \rightarrow j $\\
   $\neg j$\\
   \hline
    $\therefore \neg c $\\
  \end{tabular}\\
\end{center}

\noindent Argument is not valid. When both hypotheses are true the conclusion is false: c = True, h and j = False \\ \qed

\newpage
\noindent \textbf{Exercise 1.12.5 D}
\begin{center}
  \begin{tabular}{l}
   \text{I will buy a new car and a new house only if I get a job.}\\
   \text{I am not going to get a job.}\\
   I will buy a new house.\\
   \hline
  $ \therefore \text{I will not buy a new car.}$
  \end{tabular}
  
  \begin{center}
  c: I will buy a new car\\
  h: I will buy a new house\\
  j: I get a job\\
  \end{center}
  
   \begin{tabular}{l}
   $ j \rightarrow (c \wedge h) $\\
   $\neg j$\\
    $ h $\\
    \hline
    $\therefore \neg c $\\
  \end{tabular}\\

  
\end{center}

\noindent The argument is valid

\begin{center}
\begin{tabular}{lclcl}
   

1.& $(c \wedge h) \rightarrow j$ & Hypothesis\\
2.& $\neg (c\wedge h) \vee j$ &Conditional Identity\\
3.& $j \vee \neg (c \wedge h) $ &Commutative law, 2\\
4.& $ \neg j$ & Hypothesis\\
5.& $\neg(c \wedge h)$& Disjunctive Syllogism, 3 4\\
6.& $\neg c \vee \neg h$ & De Morgans, 5\\
7.& $\neg h \vee \neg c$ &Commutative\\
8.& $h$ & Hypothesis\\
9.& $ \neg \neg h$ & Double Negation\\
10.& $\neg c$& Disjunctive Syllogism, 7 9\\

\end{tabular}
\end{center}

\noindent \textbf{Exercise 1.13.3 B}\\

\begin{center}
  \begin{tabular}{l}
  $\exists x (P(x) \vee Q(x))$\\
  $\exists x \neg Q(x)$\\
     \hline
  $ \therefore \exists x \neg P(x) $
 

  \end{tabular}
\end{center}

\begin{center}
\begin{tabular}{|l|l|l|}
\hline
 & P& Q \\ \hline
a & F & T \\ \hline
b & F & F \\ \hline
\end{tabular}
\end{center}

\noindent When x = a$\exists x (P(x) \vee Q(x))$ is true and when x = b $\exists x \neg Q(b)$ is also true, but $\exists xP(x)$is false proving it is not valid\\
\qed 
\newpage 

\noindent \textbf{Exercise 1.13.5 D}
\begin{center}
  \begin{tabular}{l}
  Every student who missed class got a detention.\\
  Penelope is a student in the class.\\
  Penelope did not miss class.\\
    \hline
  $ \therefore \text{Penelope did not get a detention}$
  \end{tabular}
  
  \begin{center}
\begin{tabular}{l}

M(x): x missed class\\
D(x); x got a detention\\
\end{tabular}
\end{center}

\begin{center}
\begin{tabular}{l}
$\forall x (M(x) \rightarrow D(x))$\\
Penelope is a student in the class\\
$\neg M(Penelope)$\\
\hline
$\therefore \neg D(Penelope)$\\
\end{tabular}
\end{center}

\noindent The argument is invalid. If M(Penelope) is false and D(Penelope) is true, then the hypothesis are true and conclusion is false.\\
\end{center}
\qed

\noindent \textbf{Exercise 1.13.5 E}
\begin{center}
  \begin{tabular}{l}
  Every student who missed class or got a detention did not get an A.\\
  Penelope is a student in the class.\\
  Penelope got an A.\\
   \hline
  $ \therefore \text{Penelope did not get a detention}$
 \end{tabular}
\end{center}

  \begin{center}
\begin{tabular}{l}

M(x): x missed class\\
A(x): x got an A\\
D(x); x got a detention\\
\end{tabular}
\end{center}

\begin{center}
\begin{tabular}{l}
$\forall x (M(x) \vee D(x)) \rightarrow A(x)$\\
Penelope is a student in the class\\
$A(Penelope)$\\
\hline
$\therefore \neg D(Penelope)$\\
\end{tabular}
\end{center}

\begin{center}
\begin{tabular}{lclcl}
   

1.& Penelope is a student in the class & Hypothesis\\
2.& $\forall x (M(x) \vee D(x) \rightarrow \neg A(x))$ &Hypothesis\\
3.& $M(Penelope) \vee D(Penelope) \rightarrow \neg A(Penelope) $ &Universal Insatntiation\\
4.& $ A(Penelope)$ & Hypothesis\\
5.& $\neg (\neg A(Penelope))$& Double negation, 4\\
6.& $\neg (M(Penelope) \vee D(Penelope))$ & Modus Tollens, 3 5\\
7.& $\neg M(Penelope) \wedge \neg D(Penelope)$ & De Morgan\\
8.& $\neg D(Penelope) \wedge \neg M(Penelope)$ & Cummutative\\
9.& $\neg D(Penelope)$ & Simplification\\
\end{tabular}
\end{center}
\qed
\newpage
\noindent \textbf{Question 6:}\\
Exercise 2.4.1 D\\
The product of two odd integers is an odd integer.
\begin{proof}
	By Direct proof: Let x, y be odd integers. Then $\exists x | x = 2k + 1$ for some integer k and $\exists y | y = 2j + 1$ for some integer j. Then 
	\begin{align*}
	xy &= (2k+1)(2j+1)\\
	&=4kj + 2k + 2j + 1\\
	&=2(2kj + k + j) + 1\\
	\end{align*}
	Since (2kj + k + j) is an integer in $xy = 2(2kj + k +j) + 1$, the product of xy is also an odd integer.
	
\end{proof}


\noindent \textbf{Exercise 2.4.3 B}\\\\
If x is a real number and $x \leq 3, \text{then} 12 - 7x + x^2 \geq 0$.\\
\begin{proof}
	By direct proof: Let x be a real number and $ x \leq 3$
	\begin{align*}
	12-7x+x^2&= x^2 -7x + 12 \geq 0\\
	&=(x-3)(x-4) \geq 0
	\end{align*}
	Since $x \leq 3$, $x - 3 \leq 0$ and $x - 4 \leq 0$, then $(x-3)(x-4) \geq 0$ 
\end{proof}

\newpage

\noindent \textbf{Question 7:}\\

\noindent \textbf{Exercise 2.5.1 D} \\

\noindent For every integer n, if $n^2 - 2n + 7$ is even, then n is odd\\
\begin{proof}
	By Contrapositive: Assume n is an even integer such that n = 2k for some integer k. Show $n^2 - 2n +7$ is an odd integer. 
  \begin{align*}
  	n^2 - 2n + 7 &= (2k)^2 - 2(2k) + 7\\
	&=4k^2 - 4k + 7\\
	&=2(2k^2 - 2k + 3) + 1
  \end{align*}
  Since k is an integer, then $(2k^2 - 2k + 3)$ is also an integer in $2(2k^2 - 2k + 3)$. Therefore, $n^2 - 2n + 7 = 2k+1$ is an odd integer.
  
\end{proof}

\noindent \textbf{Exercise 2.5.4 A}\\

\noindent For every pair of real numbers x and y, if $x^3 + xy^2 \leq x^2y + y^3$ then $x \leq y$\\
\begin{proof}
By contrapositive: Assume for every pair of real numbers x and y, x > y. Show $x^3 + xy^2 > x^2y + y^3$\\\\

\begin{align*}
x^3 + xy^2 &= x(x^2 + y^2)\\
&> y(x^2 + y^2) \text{ by substitution x > y} \\ 
&= x^2y + y^3 \qedhere
\end{align*}
\end{proof}

\noindent \textbf{Exercise 2.5.4 B}\\

\noindent For every pair of real numbers x and y, if $x + y > 20$ then $x > 10$ or $y > 10$.\\
\begin{proof}
By Contrapositive: Assume for every pair of real numbers x and y, $x \leq 10$ and $y \leq 10$. Show $x + y \leq 20$\\
\begin{align*}
&x \leq 10 + y \leq 10\\ &= x + y \leq 20
\end{align*}
\end{proof}
  
\noindent \textbf{Exercise 2.5.5 C}\\

\noindent For every non-zero real number x, if x is irrational then $1/x$ is irrational\\
\begin{proof}
By Contrapostive: Assume x is a real number and $1/x$ is not irrational. Show that x is rational for every non zero real number. \\\\
Since $1/x$ is a real number and not irrational, then it has to be a rational number.\\
There exists an $a$ and $b$ such that $a$ and $b$ are integers.\\
\begin{align*}
&1/x = a/b \text{ ($a \not = 0$ and $b \not = 0$)}\\
&x = b/a\\
\end{align*}
Since x is a ratio of two integers, a and b, x is rational.\\ \qedhere
\end{proof}

\newpage
\noindent \textbf{Question 8:} \\


\noindent \textbf{Exercise 2.6.6 C}\\

\noindent The average of three real numbers is greater than or equal to at least one of the numbers.\\
\begin{proof}
By Contradiction: Assume x, y, z are real numbers. Show the average of the three real numbers is less than all thee numbers.
\begin{align*}
\frac{x + y + z}{3} < x,  \frac{x + y + z}{3} < y, \frac{x+y+z}{3} < z\\
\\
\frac{x+y+z}{3} + \frac{x+y+z}{3} + \frac{x+y+z}{3} &< x + y + z\\
\frac{3x + 3y + 3z}{3} &< x + y + z\\
x + y + z &< x + y + z
\end{align*}

\noindent Since $x + y + z \not < x + y + z$, then the average of three real numbers is greater than or equal to at least one of the three numbers. \qedhere
\end{proof}

\noindent \textbf{Exercise 2.6.6 D}\\

\noindent There is no smallest integer\\
\begin{proof}
By Contradiction: Assume There is a smallest integer. Let smallest integer be $k$.\\
Since k is an integer, subtracting k by 1 will give us an integer $k-1$. Since $(k - 1) < k$,  there exists a smaller integer.\\ $\therefore$ There is no smallest integer.

\end{proof}

\newpage

\noindent \textbf{Question 9:}\\
\textbf{Exercise 2.7.2 B}\\

\noindent If integers x and y have the same parity, then $x + y$ is even. The parity of a number tells whether the number is odd or even. If x and y have the same parity, they are either both even or both odd.\\

\begin{proof}
By Cases:

\noindent \textbf{Case 1:  x and y are even integers.} If x and y are even, x = 2k and y = 2j for some integer k, j\\
\begin{align*}
x + y &= 2k + 2j\\
&= 2(k + j)\\
\end{align*}
Since k and j are integers, there exists an integer $l$ such that $l = k + j$. Since $x + y = 2l$, $x + y$ is even\\\\

\noindent \textbf{Case 2:  x and y are odd integers.} If x and y are odd, x = 2m + 1and y = 2n + 1 for some integer m, n\\
\begin{align*}
x + y &= 2m + 1 + 2n + 1\\
&= 2m + 2n + 2\\
&= 2(m + n + 1)
\end{align*}
Since m, n, and 1 are integers, there exists an integer $z$, such that $z = (m + n + 1)$, Since $x + y = 2z$, x + y is even.
\end{proof}
\end{document}